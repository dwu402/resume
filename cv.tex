\documentclass[paper=a4paper,fontsize=11pt]{scrartcl}

\usepackage[english]{babel}
\usepackage[utf8x]{inputenc}
\usepackage{microtype}
\usepackage[margin=0.75in]{geometry}
\usepackage[hidelinks]{hyperref}
\usepackage{url}

\frenchspacing
\pagestyle{empty}
\renewcommand{\baselinestretch}{1.125} 


\newcommand{\pagerule}[1][2pt]{\noindent\rule{\textwidth}{#1}}
\newcommand{\usingfont}[2]{#1 #2 \par \normalsize \normalfont}

\newcommand{\sectionheader}[1]{\pagerule \vspace{0.05ex} \usingfont{\usefont{T1}{phv}{m}{v} \Large}{\noindent \hspace{-0.5em} #1} \vspace{0.5ex}}

\newlength{\spacebox}
\settowidth{\spacebox}{8888888888}

\newcommand{\indented}[1][2em]{\noindent\hangindent=#1\hangafter=0}


\begin{document}
\usingfont{\Huge \usefont{T1}{phv}{b}{n}}{\hfill David Wu}
\usingfont{\large \usefont{T1}{phv}{m}{n}}{\hfill Curriculum Vitae}

\sectionheader{Personal Details}
\vspace{-0.5em}
{\noindent\hangindent=2em\hangafter=0
\parbox{\spacebox}{\textit{Phone}}
\hspace{1.5em} (+61) 493 676 256 \par}
{\noindent\hangindent=2em\hangafter=0
\parbox{\spacebox}{\textit{Email}}
\hspace{1.5em} \href{mailto:david.jx.wu@gmail.com}{\nolinkurl{david.jx.wu@gmail.com}} \par}
{\noindent\hangindent=2em\hangafter=0
\parbox{\spacebox}{\textit{LinkedIn}}
\hspace{1.5em} \url{www.linkedin.com/in/david-jx-wu} \par}
\vspace{-0.5em}

\sectionheader{Work Experience}
\noindent \textbf{Research Fellow} \hfill
{March 2023 - Present} \par
\noindent \textit{Dept. Econometrics and Business Statistics, Monash
University, Full-time} \par
\indented \small {Modelling the surveillance and control of
hospital-acquired infections in the Victorian healthcare system using
stochastic simulation and static and temporal network analysis methods.
Additionally, organised seminars for the
\href{https://numbat.space}{NUMBAT} group, and tutored for courses in
the department on
\href{https://handbook.monash.edu/2024/units/ETC5513}{reproducible data
practices}.}
\normalsize \par\par
\noindent \textbf{Research Assistant} \hfill
{July 2020 - Feb 2023} \par
\noindent \textit{Te Pūnaha Matatini / Covid Modelling
Aotearoa, Casual} \par
\indented \small {Development of stochastic epidemic simulations on
networks in Python to assist with New Zealand Government response to
COVID-19. Implementation of a novel non-Markovian event-driven
simulation method for a system with over 5 million agents. Statistical
analysis and reporting of stochastic realisations utilising
high-performance computing.}
\normalsize \par\par
\noindent \textbf{Teaching Assistant} \hfill
{Feb 2019 - June 2022} \par
\noindent \textit{Dept. Engineering Science, University of
Auckland, Part-time} \par
\indented \small {Content development, tutoring, and administration of
undergraduate laboratory sessions on numerical methods, software
development practice, and computer systems in Python, MatLab, and C.}
\normalsize \par\par
\noindent \textbf{Software Engineer} \hfill
{Jan 2018 - Nov 2018} \par
\noindent \textit{Orion Health, Full-time} \par
\indented \small {Site reliability engineering. Automated deployment and
maintenance of Elasticsearch and Rhapsody (electronic health record
interoperability platform) in AWS. Designed and executed migration plans
for Ansible Tower and Elasticsearch instances.}
\normalsize \par\par
\noindent \textbf{Summer Student Researcher} \hfill
{Nov 2016 - Mar 2017} \par
\noindent \textit{Department of Computer Science, University of
Auckland, Full-time} \par
\indented \small {Development of computational and numerical models of a
pre-biotic replication system in Python and MatLab. Analysis and
exploration of system behaviours and parameters.}
\normalsize \par\par

\sectionheader{Education}
\noindent \textbf{PhD Engineering} \hfill
Nov 2018 - Sept 2022 \par
\noindent \textit{Dept. Engineering Science, University of
Auckland} \par
\small Thesis Topic: \emph{Computational Methods in Epidemic Simulation,
Inference and Uncertainty Quantification} \par
\indented \small Mathematical modelling of epidemics. Stochastic
simulation of large, complex systems on networks. Frequentist, Bayesian,
and likelihood-based (Fisherian) inference. Practical prediction methods
for misspecified models in mathematical epidemiology. Model
approximation approaches for inference with surrogate models.
\normalsize \par\par
\noindent \textbf{BE(Hons) Engineering Science} \hfill
Class of 2017 \par
\noindent \textit{University of Auckland} \par
\small GPA: 8.55/9.00 (A/A+ average) \par
\small Thesis Topic: \emph{Mechanistic Modelling of the Immune System's
Impact on Health} \par
\indented \small Computational and mathematical modelling methods.
Development and parameterisation of mathematical and physical models.
Continuum solid and fluid mechanics. Numerical computation methods.
Optimisation methods and data analysis. Engineering decision making and
project management.
\normalsize \par\par

\sectionheader{Skills}
\indented[1em] \parbox{\spacebox}{\textit{Languages}} 
\hspace{1.5em} English, Cantonese Chinese, Mandarin Chinese
\par
\indented[1em] \parbox{\spacebox}{\textit{Programming}} 
\hspace{1.5em} Python, bash, \LaTeX, MatLab, C++
\par
\indented[1em] \parbox{\spacebox}{\textit{Software}} 
\hspace{1.5em} AWS, MS Excel, Ansible, GIMP
\par

\sectionheader{Papers}
\vspace{-2ex}
\begin{enumerate}
\itemsep-0.25em
\item 
D. Wu, H. Petousis-Harris, J. Paynter, V. Suresh, O. J.
Maclaren, ``Likelihood-based estimation and prediction for a measles
outbreak in Samoa'' in Infectious Disease
Modelling (\href{https://doi.org/10.1016/j.idm.2023.01.007}{doi:
10.1016/j.idm.2023.01.007})
\item 
Assortment of non-peer-reviewed reports for the New Zealand Government
on COVID-19 in New Zealand, archived at
\url{https://www.covid19modelling.ac.nz/reports/}
\end{enumerate}
\vspace{-2ex}

\sectionheader{Conferences}
\vspace{0.5ex}
\noindent 
\begin{tabular}{p{1.1\spacebox} p{0.5\spacebox} p{6.7\spacebox}}
\href{https://www.mathematics.org.au/sys/pages/plain.php?page\_id=39\&conf\_id=61}{\textbf{ANZIAM}} & 2024 & Contributed
talk: ``Temporal trends of hospital transfer networks in Victoria for
controlling the spread of antibiotic resistance'' \\
\href{https://www.elsevier.com/events/conferences/all/international-conference-on-infectious-disease-dynamics}{\textbf{Epidemics
9}} & 2023 & Contributed poster: ``Estimation of Network Epidemic Models
using Surrogate Correction'' \\
\href{https://ecmtb2022.org}{\textbf{ECMTB}} & 2022 & Contributed
poster: ``Sneaking non-Markovian dynamics into Gillespie's direct method
for epidemic simulation'' \\
\textbf{NZWUQIP} & 2021 & Contributed talk: ``Likelihood-based
estimation and prediction for misspecified epidemic models: an
application to measles in Samoa'' \\
\href{http://www.maths.mq.edu.au/ANZIAM2020}{\textbf{ANZIAM}} & 2020 & Contributed
talk: ``Infectious disease outbreaks: inference and prediction under
model misspecification and partially observed data'' \\
\href{https:/minz.org.nz/2019/}{\textbf{MINZ}} & 2019 & Student
Moderator, Challenge 4: ``How can Mercury improve the generation
efficiency of the Waikato hydro scheme?'' \\
\href{http://conferences.science.unsw.edu.au/SMB2018/}{\textbf{SMB}} & 2018 & Contributed
talk: ``A dynamical system model of host-pathogen interaction
illustrates the role of the immune system in resilience to
infection'' \\
\end{tabular}

\sectionheader{Awards and Honours}
\vspace{0.5ex}\noindent
\begin{tabular}{p{1.0\spacebox} p{0.2\spacebox} p{6.5\spacebox}}
2023 & \multicolumn{2}{l}{2nd Place, UN Datathon (Down Under Data
Wizards team)} \\
& & \small{A data science competition focused on progressing the UN
Sustainable Development Goals, approximately 150 entries.}\normalsize\\ 
2020 & \multicolumn{2}{l}{Prime Minister's Science Prize (Te Pūnaha
Matatini COVID-19 group)} \\
& & \small{Awarded annually for a transformative scientific discovery or
achievement, which has had a significant economic, health, social,
and/or environmental impact on New Zealand, or
internationally.}\normalsize\\ 
2018 & \multicolumn{2}{l}{University of Auckland Doctoral
Scholarship} \\
& & \small{Awarded to high-achieving doctoral candidates (GPA 8.0 or
above) applying for admission to an approved doctoral program at the
University of Auckland.}\normalsize\\ 
2015-2017 & \multicolumn{2}{l}{University of Auckland Faculty of
Engineering Dean's Honours List} \\
& & \small{Awarded annually to students who have demonstrated excellence
in academic performance by being in the top 5\% of their year of
study.}\normalsize\\ 
\end{tabular}

\end{document}